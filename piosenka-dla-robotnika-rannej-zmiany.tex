%w Godzina słynna: piąta pięć
\documentclass[a5paper]{article}
 \usepackage[english,bulgarian,russian,ukrainian,polish]{babel}
 \usepackage[utf8]{inputenc}
 %\usepackage{polski}
 \usepackage[T1]{fontenc}
 \usepackage[margin=1.5cm]{geometry}
 \usepackage{multicol}
 \setlength\columnsep{10pt}
 \begin{document}
 %\pagenumbering{gobble}


\noindent
\fontsize{12pt}{15pt}\selectfont
\textbf{Piosenka dla robotnika rannej zmiany} \\
\fontsize{8pt}{10pt}\selectfont
Stare Dobre Małżeństwo / sł. Edward Stachura, muz. Krzysztof Myszkowski \\ \\
\fontsize{10pt}{12pt}\selectfont
\leftskip0cm
\begin{tabular}{@{}p{7.50cm}p{3cm}@{}}
\noindent
Godzina słynna: piąta pięć & E \\
Naciska budzik, dźwiga się & E \\
Do kuchni drogę zna na pamięć & E \\
Prowadzą go tam nogi same & E E7 \\
Pod kran pakuje śpiący łeb & A \\
Przez chwilę jeszcze śpi jak w łóżku & E \\
Dopóki nie posłyszy plusku & H7 \\
I wtedy wreszcie budzi się & A E H7 \\ \\
\end{tabular}

\leftskip1cm
\noindent
\begin{tabular}{@{}p{6.50cm}p{3cm}@{}}
Aniele Pracy - stróżu mój \\
Jak ciężki robotnika znój \\
Zbożowa kawa, smalec, chleb \\
Salceson czasem, kiedy jest \\
Do teczki drugie pcha śniadanie \\
I teraz szybko na przystanek \\
W tramwaju tłok i nie ma Boga \\
Jest ramię w ramię, w nogę noga \\
Kimanie na stojąco jest \\ \\

& E E A E H7 A E H7 \\ \\
\end{tabular}

\leftskip1cm
\noindent
\begin{tabular}{@{}p{6.50cm}p{3cm}@{}}
Aniele Pracy - stróżu mój \\
Jak ciężki robotnika znój \\
Przez osiem godzin praca wre \\
Jak z bicza strzelił minął dzień \\
Już w domu siedzi przed ekranem \\
Na stole flaszka z marcepanem \\
Dziś chłopcy grają ważny mecz \\
Przez cały czas w ataku nasi \\ \\
\end{tabular}

\leftskip1cm
\noindent
\begin{tabular}{@{}p{6.50cm}p{3cm}@{}}
Aniele Pracy - stróżu mój \\
Jak ciężki robotnika znój \\
Niech nas ukoi dobry sen \\
Najlepsza w końcu jest to rzecz \\
I co się śni? \\
Podwyżka cen \\
Aniele Pracy - stróżu mój \\
Jak ciężki robotnika znój \\ \\
\end{tabular}

\end{document}
