%w Zagram dla Ciebie na każdej gitarze świata
%r Graj nam, graj pieśni skrzydlata
\documentclass[a5paper]{article}
 \usepackage[english,bulgarian,russian,ukrainian,polish]{babel}
 \usepackage[utf8]{inputenc}
 %\usepackage{polski}
 \usepackage[T1]{fontenc}
 \usepackage[margin=1.5cm]{geometry}
 \usepackage{multicol}
 \setlength\columnsep{10pt}
 \begin{document}
 %\pagenumbering{gobble}


\noindent
\fontsize{12pt}{15pt}\selectfont
\textbf{Piosenka wiosenna} \\
\fontsize{8pt}{10pt}\selectfont
Wolna Grupa Bukowina / sł. Wojtek Bellon, muz. Wojciech Jarociński \\ \\
\fontsize{10pt}{12pt}\selectfont
\leftskip0cm
\begin{tabular}{@{}p{8.50cm}p{3cm}@{}}
\noindent
into & G D C / G D C / G h \\ \\
Zagram dla Ciebie na każdej gitarze świata & e h C D \\
Na ulic fletach, na nitkach babiego lata & e h C D \\
Wyśpiewam, jak potrafię księżyce na rozstajach & G D G C \\ \\
I wrześnie, i stycznie, i maje & h C a D \\
I zagubione dźwięki, barwy na płótnach Vlamincka & h h C D (D4)\\
I słońce wędrujące promienia ścieżynką & h C a7 D \\ \\
\end{tabular}

\leftskip1cm
\noindent
\begin{tabular}{@{}p{7.50cm}p{3cm}@{}}
Graj nam, graj pieśni skrzydlata & G D C9/5 G \\
Wiosna, taniec nasz niesie po łąkach & h C a7 D \\
Zatańczymy się w sobie do lata & G D G D C9/5 \\
Zatańczymy się w siebie bez końca & G D4 C9/5 D G \\
& G D C \\ \\
\end{tabular}

\leftskip0cm
\noindent
\begin{tabular}{@{}p{8.50cm}p{3cm}@{}}
A blask, co oswietla me ręce, gdy piszę \\
Nabrzmiał potrzebą rozerwania ciszy \\
Przez okno wyciekł, pełna go teraz chmara wronia \\ \\
Dzobi się w dziobów końcach, a w ogonach ogoni \\ 
A pieśń moja to niknie, to wraca \\
I nie wiem, co bym zrobił, gdybym ją utracił \\ \\
\end{tabular}

\leftskip1cm
\noindent
\begin{tabular}{@{}p{8.00cm}@{}}
Graj nam, graj pieśni skrzydlata…
\end{tabular}

\end{document}
