%w Nikt nie zna ścieżek gwiazd
\documentclass[a5paper]{article}
 \usepackage[english,bulgarian,russian,ukrainian,polish]{babel}
 \usepackage[utf8]{inputenc}
 %\usepackage{polski}
 \usepackage[T1]{fontenc}
 \usepackage[margin=1.5cm]{geometry}
 \usepackage{multicol}
 \setlength\columnsep{10pt}
 \begin{document}
 %\pagenumbering{gobble}


\noindent
\fontsize{12pt}{15pt}\selectfont
\textbf{Kim właściwie była ta piękna pani} \\
\fontsize{8pt}{10pt}\selectfont
Stare Dobre Małżeństwo \\ \\
\fontsize{10pt}{12pt}\selectfont
\leftskip0cm
\begin{tabular}{@{}p{8.50cm}p{3cm}@{}}
\noindent
Nikt nie zna ścieżek gwiazd, & a G \\
Wybrańcem kto wśród nas? & e a \\
Zapukał ktoś - to do mnie gość!? & d C G \\
Włóczyłem się jak cień, & a G \\
Czekałem na ten dzień & e a \\
Już stoisz w drzwiach jak dziwny ptak. & d C G \\ \\

Więc bardzo proszę wejdź, & F G \\
Tu siadaj, rozgość się & e a \\
I zdrać mi, kim tyś jest, Madame? & F G \\
Albo nie zdradzaj mi, & e a \\
Lepiej nie mówmy nic, & G \\
Lepiej nie mówmy nic. & F C \\ \\

Nieśmiało sunie brzask, & a G \\
Zatrzymać chciałbym czas & e a\\
Inaczej jest - czas musi biec. & d C G\\
Gdzieś w dali zapiał kur, & a G\\
Niemodny wdziewasz strój & e a\\
Już stoisz w drzwiach jak dziwny ptak. & d C G\\ \\

Więc jednak musisz pójść & F G \\
Posyłasz mi przez próg & e a \\
Ulotny uśmiech twój, Madame & F G \\
Lecz będę czekać, przyjdź! & e a \\
Gdy tylko zechcesz, przyjdź & G \\
Będziemy razem żyć & F a \\
Ja będę czekać, przyjdź & e a \\
Gdy tylko zechcesz, przyjdź	& G \\
Będziemy razem żyć.	& F C
\end{tabular}

\end{document}
