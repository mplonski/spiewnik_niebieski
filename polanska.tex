%w Słońce ma się już ku zachodowi
%r Jeszcze chwile, jeszcze dwie, i znów stanę u stóp raju
\documentclass[a5paper]{article}
 \usepackage[english,bulgarian,russian,ukrainian,polish]{babel}
 \usepackage[utf8]{inputenc}
 %\usepackage{polski}
 \usepackage[T1]{fontenc}
 \usepackage[margin=1.5cm]{geometry}
 \usepackage{multicol}
 \setlength\columnsep{10pt}
 \begin{document}
 %\pagenumbering{gobble}


\noindent
\fontsize{12pt}{15pt}\selectfont
\textbf{Polańska} \\
\fontsize{8pt}{10pt}\selectfont
Dom o Zielonych Progach / sł. i muz. Wojtek Szymański \\ \\
\fontsize{10pt}{12pt}\selectfont
\leftskip0cm
\begin{tabular}{@{}p{7.50cm}p{3cm}@{}}
\noindent
Słońce ma się już ku zachodowi & G C \\
Czerwienią góry maluje & C D \\
Przez Biskupi Łan podąża & G C \\
Brodząc w trawy po szyję & C D \\
Zanurzam się w dolinę & h C \\
Wdycham żywiczny zapach lasu & h C \\
Jeszcze strumień, jeszcze potok & a7 C \\
I będę znów u siebie w domu & a7 C D \\ \\
\end{tabular}

\leftskip1cm
\noindent
\begin{tabular}{@{}p{6.50cm}p{3cm}@{}}
Jeszcze chwila, jeszcze dwie & G C G \\
I znów stanę u stóp raju & C D \\
Przed Polańską gdzie we mgle & G C G \\
Gniade konie skubią trawę & a7 C D \\ \\
\end{tabular}

\leftskip0cm
\noindent
\begin{tabular}{@{}p{7.50cm}@{}}
Śpiew Łemków po dolinie \\
Wiatr wędrownik niesie \\
A z dzwonnicy się dobywa \\
Bicie starych dzwonów \\
We mgle jakby widać \\
Pstrych chat białe kształty \\
Tylko ludzie gdzieś zniknęli \\
Gdzieś odeszli stąd daleko \\ \\
\end{tabular}

\leftskip1cm
\noindent
\begin{tabular}{@{}p{6.50cm}@{}}
Jeszcze chwila, jeszcze dwie… \\ \\
\end{tabular}

\leftskip0cm
\noindent
\begin{tabular}{@{}p{7.50cm}@{}}
Takie tu czasy, takie czasy \\
Że zegar jakby stanął w miejscu \\
Ludzie życzliwi są i zawsze będą \\
A domy malowane jakby pędzlem Boga \\
A ponad wszystkim króluje Polańska \\
Rozkołysana wśród morza mgieł \\
Woła mnie dziś pieśnią ku sobie \\
Może odnajdę tu moje marzenia
\end{tabular}

\end{document}
