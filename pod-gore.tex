%w Gdy usłyszę pierwszy lodów trzask
%r I pod górę, tam gdzie wieje wiatr
\documentclass[a5paper]{article}
 \usepackage[english,bulgarian,russian,ukrainian,polish]{babel}
 \usepackage[utf8]{inputenc}
 %\usepackage{polski}
 \usepackage[T1]{fontenc}
 \usepackage[margin=1.5cm]{geometry}
 \usepackage{multicol}
 \setlength\columnsep{10pt}
 \begin{document}
 %\pagenumbering{gobble}


\noindent
\fontsize{12pt}{15pt}\selectfont
\textbf{Pod górę} \\
\fontsize{8pt}{10pt}\selectfont
Zgórmysyny / sł. i muz. Tomek Jarmużewski \\ \\
\fontsize{10pt}{12pt}\selectfont
\leftskip0cm
\begin{tabular}{@{}p{9.5cm}p{3cm}@{}}
\noindent
Gdy usłyszę pierwszy lodów trzask, & e D C \\
Kiedy pierwsze ciepłe tchnienie wyda ziemia, & e D C \\
Wtedy każdym nerwem czuję, że najwyższy nadszedł czas, & e D C \\
Zbudzić się z zimowego otępienia. & a D \\ \\

Gdy spod śniegu pierwszy wzejdzie kwiat, \\
Snom i drzewom zacznie znów ubywać cienia, \\
Znajdę drogę, którą pójdę przez zalany słońcem świat, \\
Tak jak rusza woda w dół strumienia. \\ \\
\end{tabular}

\leftskip1cm
\noindent
\begin{tabular}{@{}p{8.5cm}p{3cm}@{}}
I pod górę, tam gdzie wieje wiatr. & G D C \\
Niech prowadzą moje marzenia. & e D C \\
Nie potrzeba wcale świata zmieniać, & e D C \\
Kiedy z życia można tyle brać. & a D \\
					& a D G \\ \\
\end{tabular}

\leftskip0cm
\noindent
\begin{tabular}{@{}p{8.5cm}p{3cm}@{}}
Trzeba iść i innym szczęście dać, \\
Odwzajemnić wszystko to, co było dane, \\
Spijać radość, którą niesie każdy wieczór, każdy brzask, \\
Moc pragnień niech się czynem stanie. \\ \\
\end{tabular}

\leftskip1cm
\noindent
\begin{tabular}{@{}p{8.5cm}p{3cm}@{}}
	I pod górę, tam gdzie wieje wiatr… \\ \\
	& E \\
	I pod górę, tam gdzie wieje wiatr & A E D \\
	Niech prowadzą moje marzenia. & fis E D \\
	Nie potrzeba wcale świata zmieniać, & fis E D \\
	Kiedy z życia można tyle brać. & h E A \\ \\
\end{tabular}

\end{document}
