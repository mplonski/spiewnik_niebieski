%w Powrócę kiedyś w miast ciemny gotyk
%r Znów będę kłaniał się starym gazowym latarniom
\documentclass[a5paper]{article}
 \usepackage[english,bulgarian,russian,ukrainian,polish]{babel}
 \usepackage[utf8]{inputenc}
 %\usepackage{polski}
 \usepackage[T1]{fontenc}
 \usepackage[margin=1.5cm]{geometry}
 \usepackage{multicol}
 \setlength\columnsep{10pt}
 \begin{document}
 %\pagenumbering{gobble}


\noindent
\fontsize{12pt}{15pt}\selectfont
\textbf{Gazowe latarnie} \\
\fontsize{8pt}{10pt}\selectfont
sł. i muz. Marek Majewski \\ \\
\fontsize{10pt}{12pt}\selectfont
\leftskip0cm
\begin{tabular}{@{}p{9.00cm}p{3cm}@{}}
\noindent
Znów będę kłaniał się starym gazowym latarniom & C F G \\
Zanucę drzewom odpowiedź na szum powitania & C F G \\
Zjarzę do okien witrynom i nocnym kawiarniom & C F G \\
A może zebry nakarmię na skrzyżowaniach & C F G \\ \\
\end{tabular}

\leftskip1cm
\noindent
\begin{tabular}{@{}p{8.00cm}p{3cm}@{}}
Powrócę kiedyś w miast ciemny gotyk & C e C7 \\
Z gór szerokich jasnych łąk & F G C \\
Aby zrozumieć mowę tęsknoty & C e C7 \\
Aby odetchnąć ciepłem twoim rąk & F G C \\
Zabiorę z sobą zapach słońca & E7 a \\
I blask poranka roześmiany & F G C \\
I koncert na sześć strun bez końca & E7 a \\
Na pieciolinii zapisany. & F G \\ \\
\end{tabular}

\leftskip1cm
\noindent
\begin{tabular}{@{}p{8.00cm}p{3cm}@{}}
Znów będę… \\ \\
\end{tabular}

\leftskip0cm
\noindent
\begin{tabular}{@{}p{9.00cm}@{}}
Pokażę ci jak pachną latarnie \\
Jak się noc zanurza w rzeki czerń \\
A kiedy jasna mgła nas ogarnie \\
Pierwszy tramwaj nam przyniesie dzień \\
Przez barokowy park pójdziemy \\
Zanim do domu powrócimy \\
Będziemy liczyć dni jesieni \\
I aby do zimy, aby do zimy… \\ \\
\end{tabular}

\leftskip1cm
\noindent
\begin{tabular}{@{}p{8.00cm}@{}}
Znów będę…
\end{tabular}

\end{document}
