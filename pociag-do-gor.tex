%w Dokąd pędzą moje myśli
%r Bo ja mam pociąg do gór
\documentclass[a5paper]{article}
 \usepackage[english,bulgarian,russian,ukrainian,polish]{babel}
 \usepackage[utf8]{inputenc}
 %\usepackage{polski}
 \usepackage[T1]{fontenc}
 \usepackage[margin=1.5cm]{geometry}
 \usepackage{multicol}
 \setlength\columnsep{10pt}
 \begin{document}
 %\pagenumbering{gobble}


\noindent
\fontsize{12pt}{15pt}\selectfont
\textbf{Pociąg do gór} \\
\fontsize{8pt}{10pt}\selectfont
Pod Wiatr / sł. i muz. Radosław Wójtowicz\\ \\
\fontsize{10pt}{12pt}\selectfont
\leftskip0cm
\begin{tabular}{@{}p{8.50cm}p{3cm}@{}}
\noindent
Dokąd pędzą moje myśli & G D \\
Gdy się kończy sen & C G \\
Czy jak zwykle znów się przyśni & e A \\
Kamienista perć & D D7  \\ \\

Czemu serce się pakuje & G H7 \\
I jak pociąg gna & e A \\
Czemu tory mnie prowadzą & C D \\
Zawsze właśnie tam & G D \\ \\
\end{tabular}

\leftskip1cm
\noindent
\begin{tabular}{@{}p{7.50cm}p{3cm}@{}}
Bo ja mam pociąg do gór & G - G D \\
Codziennie o siódmej rano & a C \\
I zanim się obudzę na szlak zabiera mnie & a C D D \\
 
Para bucha spod kół & G - G D \\
A dróżnik przed szlabanem & a C \\
Wycina w jej obłokach witraże moich snów & a D G D \\ \\
\end{tabular}

\leftskip0cm
\noindent
\begin{tabular}{@{}p{7.50cm}p{3cm}@{}}
A w wagonie babie lato \\
Srebrne nici tka \\
I na czemplikowym smyczku \\
Na dwa głosy gra \\ \\
 
W siwej brodzie Ziemianina \\
Wplątuje się wiersz \\
O marzeniach, z których jedno \\
Co noc spełnia się \\ \\
\end{tabular}

\leftskip1cm
\noindent
\begin{tabular}{@{}p{7.50cm}p{3cm}@{}}
Bo ja mam pociąg do gór… \\ \\
\end{tabular}

\leftskip0cm
\noindent
\begin{tabular}{@{}p{7.50cm}p{3cm}@{}}
Połoniny już się śmieją \\
Do okiennych szyb \\
Moje palce wystukują \\
Niecierpliwy rytm \\ \\

W sercu gór końcowa stacja \\
Gwizd rozlega się \\
I jak zwykle gwizd pociągu \\
To budzika dźwięk \\ \\
\end{tabular}

\leftskip1cm
\noindent
\begin{tabular}{@{}p{7.50cm}p{3cm}@{}}
Bo ja mam pociąg do gór… \\ \\
\end{tabular}

\end{document}
