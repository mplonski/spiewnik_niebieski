%w A jeżeli wiosną przejdę nad głębiną
%r I nic więcej, bo przecież nad taką równiną
\documentclass[a5paper]{article}
 \usepackage[english,bulgarian,russian,ukrainian,polish]{babel}
 \usepackage[utf8]{inputenc}
 %\usepackage{polski}
 \usepackage[T1]{fontenc}
 \usepackage[margin=1.5cm]{geometry}
 \usepackage{multicol}
 \setlength\columnsep{10pt}
 \begin{document}
 %\pagenumbering{gobble}


\noindent
\fontsize{12pt}{15pt}\selectfont
\textbf{Piosenka mazowiecka (Nostalgia) } \\
\fontsize{8pt}{10pt}\selectfont
sł. i muz. Iwona Piastowska \\ \\
\fontsize{10pt}{12pt}\selectfont
\leftskip0cm
\begin{tabular}{@{}p{8.50cm}p{3cm}@{}}
\noindent
A jeżeli przyjdę wiosną nad głębinę & D A \\
A bodaj się wszyscy śmieli, pewnie gdzieś popłynę & e H7 e \\
Między wierzby krzywe, co się słońcem mienią \\
Albo wyspy wikliny szumiące zielenią. \\ \\

Zaraz się zachwycę, zaraz się zasłucham \\
W jakieś bajdy zaplątane w barokowych łukach \\
Swoje dzieje spiszę na piaszczystym trakcie,\\
Wyślę pocztą gołębią cztery mile za piec.\\ \\
\end{tabular}

\leftskip1cm
\noindent
\begin{tabular}{@{}p{7.50cm}p{3cm}@{}}
        Nic więcej, bo przecież nad taką równiną & D Fis \\
        No jak tu nie lecieć, no jak tu nie płynąć & h E A \\
        Od końca do końca, na teraz na teraz i zawsze \\
        No jak tu nie słuchać, no jak tu nie patrzeć. \\ \\
\end{tabular}

\leftskip0cm
\noindent
\begin{tabular}{@{}p{7.50cm}p{3cm}@{}}
A gdy stąd odejdę zasmuconym sadem \\
Będę długo spać jesienią w kącie pod obrazem\\
A gdy się obudzę, choćby mróz był wielki\\
Wrócę przecież jak inni drugim brzegiem rzeki.\\ \\
\end{tabular}

\leftskip1cm
\noindent
\begin{tabular}{@{}p{7.50cm}p{3cm}@{}}
        Nic więcej… | x2
\end{tabular}

\end{document}
