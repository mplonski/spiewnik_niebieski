%w W kołnierz wtulam twarz
%r Gdy koncert grać ten na trąbki i skrzypce
\documentclass[a5paper]{article}
 \usepackage[english,bulgarian,russian,ukrainian,polish]{babel}
 \usepackage[utf8]{inputenc}
 %\usepackage{polski}
 \usepackage[T1]{fontenc}
 \usepackage[margin=1.5cm]{geometry}
 \usepackage{multicol}
 \setlength\columnsep{10pt}
 \begin{document}
 %\pagenumbering{gobble}


\noindent
\fontsize{12pt}{15pt}\selectfont
\textbf{Koncert} \\
\fontsize{8pt}{10pt}\selectfont
Stare Dobre Małżeństwo \\ \\
\fontsize{10pt}{12pt}\selectfont
\leftskip0cm
\begin{tabular}{@{}p{8.50cm}p{3cm}@{}}
\noindent
W kołnierz wtulam twarz & H7 e \\
Chowam się przed miastem & H7 e \\
Jego cienie zlobia w mojej twarzy wąwóz & C G H7 \\
Trzeszczy jak ułamek szkła & C G \\
Mój codzienny niepokój & a G \\
Jak wydostać się z cienia & C G \\
Może wtedy & H7 \\ \\
\end{tabular}

\leftskip1cm
\noindent
\begin{tabular}{@{}p{7.50cm}p{3cm}@{}}
Gdyby koncert grać & C \\
Ten na trąbki i skrzypce & G \\
Tak, by dźwięki ułożyły się w wiersz & C G \\
Gdyby łyżką światła & C \\
Rozweselić to wszystko & G \\
Żeby we mnie zaśpiewało coś też & H7 e \\ \\
\end{tabular}

\leftskip0cm
\noindent
\begin{tabular}{@{}p{7.50cm}p{3cm}@{}}
Rośnie we mnie mgła \\
Jak ze studzien stu \\
Nie wiem, ilu trzeba ksiąg, by ją rozwiać \\
Jedno wiem, że muszę biec \\
Póki sił mi wystarczy \\
Póki tylko ta nuta… \\
Mam ją w sobie! \\ \\
\end{tabular}

\leftskip1cm
\noindent
\begin{tabular}{@{}p{7.50cm}p{3cm}@{}}
Będę koncert grać \\
Ten na trąbki i skrzypce \\
Tak, by dźwięki ułożyły się w wiersz \\
Będę łyżką światła \\
Rozweselać to wszystko \\
Żeby w tobie zaśpiewało coś też 
\end{tabular}

\end{document}
