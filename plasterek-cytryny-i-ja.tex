%w Na koniec ciężkiego dnia
%r Od brzegu szklanego, po szary horyzont
\documentclass[a5paper]{article}
 \usepackage[english,bulgarian,russian,ukrainian,polish]{babel}
 \usepackage[utf8]{inputenc}
 %\usepackage{polski}
 \usepackage[T1]{fontenc}
 \usepackage[margin=1.5cm]{geometry}
 \usepackage{multicol}
 \setlength\columnsep{10pt}
 \begin{document}
 %\pagenumbering{gobble}


\noindent
\fontsize{12pt}{15pt}\selectfont
\textbf{Plasterek cytryny i ja} \\
\fontsize{8pt}{10pt}\selectfont
Andrzej Korycki, Dominika Żukowska / sł. i muz. Andrzej Korycki \\ \\
\fontsize{10pt}{12pt}\selectfont
\leftskip0cm
\begin{tabular}{@{}p{8.50cm}p{3cm}@{}}
\noindent
intro & e e C H7 \\ \\
Na koniec ciężkiego dnia, na koniec ciężkiego dnia & e e \\
Gdy słońce już w lesie zaczyna się chować. & e a \\
To nie wiem jak wy, ale ja, to nie wiem jak wy, ale ja & a a \\
Ja lubię tak sobie ot ciut… pożeglować. & H7 C7 H7 e \\ \\

Szklaneczka cieszy się, bo; szklaneczka cieszy się, bo & e e \\
Bo widzi, że Coli butelkę wytaczam. & e a \\
I przypomina mi o…, i przypomina mi o… & a a \\
O odrobinie rudego… whiskacza. & H7 C7 H7 e \\ \\
\end{tabular}

\leftskip1cm
\noindent
\begin{tabular}{@{}p{7.50cm}p{3cm}@{}}
Od brzegu szklanego, po szklany horyzont & e a \\
Gdzie szronu rozpina się mgła, & C H7 e H7 \\
Płyniemy spokojnie, pośpiechem się brzydząc, & e a \\
Plasterek cytryny i ja. & C H7 e H7 \\ \\
\end{tabular}

\leftskip0cm
\noindent
\begin{tabular}{@{}p{11.50cm}@{}}
Nie raz słyszałem już, że; nie raz słyszałem już, że \\
Że takie wieczorne pływanie mnie zgubi. \\
A ja właśnie taki mam styl, a ja właśnie taki mam styl \\
Niech każdy tak sobie żegluje – jak lubi. \\ \\

Lecz jeśli martwi was fakt, lecz jeśli martwi was fakt, \\
Że rejsy zaczynać wieczorem najtrudniej. \\
W porządku-przyrzekam wam dziś, w porządku-przyrzekam wam dziś, \\
Od jutra już zacznę wypływać … w południe \\ \\

2. gitara (7 próg) \\
Zwrotka: a / d / E7 F7 E7 a \\
Refren: a / d / F E a E
\end{tabular}

\end{document}
