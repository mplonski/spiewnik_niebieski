%w Całe życie w niebo idzie
%r Jak ciała nasze w mrocznym rytmie
\documentclass[a5paper]{article}
 \usepackage[english,bulgarian,russian,ukrainian,polish]{babel}
 \usepackage[utf8]{inputenc}
 %\usepackage{polski}
 \usepackage[T1]{fontenc}
 \usepackage[margin=1.5cm]{geometry}
 \usepackage{multicol}
 \setlength\columnsep{10pt}
 \begin{document}
 %\pagenumbering{gobble}


\noindent
\fontsize{12pt}{15pt}\selectfont
\textbf{Pieśń XXIX} \\
\fontsize{8pt}{10pt}\selectfont
Dom o Zielonych Progach / sł. Jerzy Harasymowicz, muz. Wojciech Szymański \\ \\
\fontsize{10pt}{12pt}\selectfont
\leftskip0cm
\begin{tabular}{@{}p{8.00cm}p{3cm}@{}}
\noindent
D C G D x2 & \\ \\
Całe życie w niebo idzie & D2 \\
Mój połoniński pochód & C7+ \\
I buki - srebrni jeźdźcy & G6 \\
Nad nimi wiosny sokół & D2 \\ \\
I nadał tamtej połoniny wiatr & \\
I chmur wiosennych grzywy & \\
I na chorągwi wspomnień twarz & \\
Z włosami wiejącymi & \\ \\
\end{tabular}

\leftskip1cm
\noindent
\begin{tabular}{@{}p{7.00cm}p{3cm}@{}}
Jak ciała nasze w mrocznym rytmie & D \\
Wznosiły się góry opadały & e \\
Tak dzieje się gdy wiosna przyjdzie & G \\
Wypala miłość stare trawy & D \\ \\
D C G D x2 & \\ \\
\end{tabular}

\leftskip0cm
\noindent
\begin{tabular}{@{}p{8.00cm}@{}}
Całe życie w niebo idzie \\
Mój połoniński pochód \\
I buki - srebrni jeźdźcy \\
Nad nimi wiosny sokół \\ \\
Jak popiół rozwiały się grzechy \\
W ciszy ktoś zawilce zasiał \\
I tylko grzmią włosy przestrzeni \\
W wielkich oknach mego świata \\ \\
\end{tabular}

\leftskip1cm
\noindent
\begin{tabular}{@{}p{7.00cm}@{}}
Jak ciała nasze w mrocznym rytmie…
\end{tabular}

\end{document}
