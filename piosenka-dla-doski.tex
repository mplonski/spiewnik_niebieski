%w Jeszcze noc, jeszcze cienie się chwieją
%r To będzie dobry dzień jeśli rano mnie przywitasz
\documentclass[a5paper]{article}
 \usepackage[english,bulgarian,russian,ukrainian,polish]{babel}
 \usepackage[utf8]{inputenc}
 %\usepackage{polski}
 \usepackage[T1]{fontenc}
 \usepackage[margin=1.5cm]{geometry}
 \usepackage{multicol}
 \setlength\columnsep{10pt}
 \begin{document}
 %\pagenumbering{gobble}


\noindent
\fontsize{12pt}{15pt}\selectfont
\textbf{Piosenka dla Dośki} \\
\fontsize{8pt}{10pt}\selectfont
Majstrowie Gór / sł. i muz. Jarek Kochanowski \\ \\
\fontsize{10pt}{12pt}\selectfont
\leftskip0cm
\begin{tabular}{@{}p{7.50cm}p{3cm}@{}}
\noindent
\emph{kapodaster II próg} \\ \\

Jeszcze noc, jeszcze cienie się chwieją & G6 C9 G6 C9 \\
Cichym snem oddycha cały dom & G6 C9 e D \\
Moje myśli błądzą między złudą i nadzieją & G6 C9 G6 C9 \\
A czasami do snów Twoich zajrzeć chcą & G6 C9/D G/G7 \\ \\
\end{tabular}

\leftskip1cm
\noindent
\begin{tabular}{@{}p{6.50cm}p{3cm}@{}}
To będzie dobry dzień & C D \\
Jeśli rano mnie przywitasz & G C \\
Ciepłych rąk dotknięciem & e \\
Bez niepotrzebnych słów & A D/D7 \\
To będzie dobry dzień & C D  \\
Z Twojej twarzy to wyczytam & G C \\
Dzień się budzi, Ty się budzisz & G C \\
Witaj dniu (bis) & D G (G7) \\ \\
\end{tabular}
  
\leftskip0cm
\noindent
\begin{tabular}{@{}p{7.50cm}p{3cm}@{}}
Jakie imię chcesz darować mu na drogę \\
Jakim słowem przywita go nasz dom \\
Czy gdy jesień przyjdzie będzie jej za osłodę \\
Wszak we dwoje jesień nie jest porą złą \\ \\
\end{tabular}

\leftskip1cm
\noindent
\begin{tabular}{@{}p{7.50cm}p{3cm}@{}}
To będzie dobry dzień…
\end{tabular}

\end{document}
