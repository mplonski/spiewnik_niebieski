%w A ja wolę na zielonej łące siedzieć
\documentclass[a5paper]{article}
 \usepackage[english,bulgarian,russian,ukrainian,polish]{babel}
 \usepackage[utf8]{inputenc}
 %\usepackage{polski}
 \usepackage[T1]{fontenc}
 \usepackage[margin=1.5cm]{geometry}
 \usepackage{multicol}
 \setlength\columnsep{10pt}
 \begin{document}
 %\pagenumbering{gobble}


\noindent
\fontsize{12pt}{15pt}\selectfont
\textbf{Pójdę w połoniny (A ja wolę…)} \\
\fontsize{8pt}{10pt}\selectfont
Dom o Zielonych Progach / sł. i muz. Wojtek Szymański \\ \\
\fontsize{10pt}{12pt}\selectfont
\leftskip0cm
\begin{tabular}{@{}p{6.50cm}p{3cm}@{}}
\noindent
A ja wolę & E a \\
Na zielonej łące siedzieć & E a G F \\
Niźli w szarym mieście & C \\
Które głuche jest & E \\
Na moje wołanie & F C \\
Na mój niemy krzyk & F C \\
Na moją samotność & F C \\
Obojętne jest & E a G F C C \\ \\
\end{tabular}

\leftskip1cm
\noindent
\begin{tabular}{@{}p{5.50cm}p{3cm}@{}}
Pójdę w połoniny \\
W roztańczone bujne trawy \\
Pod rękę razem \\
Z polnym wiatrem \\
Między szumem liści \\
Ukryte słowa dla mnie \\
Zanucę je głośniej \\
Niech popłyną dalej gdzieś \\ \\
\end{tabular}

\leftskip0cm
\noindent
\begin{tabular}{@{}p{6.50cm}@{}}
Niech na moim niebie \\
Rozniebieszczą się gwiazdy \\
Niczym Mały Książę \\
Będę sobie szedł \\
Może spotkam różę \\
Której kolców brak \\
Która zdradzi mi swe imię \\
I swój uśmiech da \\ \\

Nananananajnaj... & E a E a G F C E F C F C F C E a G F C \\ \\
\end{tabular}

\leftskip1cm
\noindent
\begin{tabular}{@{}p{5.50cm}@{}}
Pójdę w połoniny…
\end{tabular}

\end{document}
