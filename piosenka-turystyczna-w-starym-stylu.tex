%w Poniedziałek trąci nudą
%r A w sobotę, a w sobotę, wezmę plecak i gitarę
\documentclass[a5paper]{article}
 \usepackage[english,bulgarian,russian,ukrainian,polish]{babel}
 \usepackage[utf8]{inputenc}
 %\usepackage{polski}
 \usepackage[T1]{fontenc}
 \usepackage[margin=1.5cm]{geometry}
 \usepackage{multicol}
 \setlength\columnsep{10pt}
 \begin{document}
 %\pagenumbering{gobble}


\noindent
\fontsize{12pt}{15pt}\selectfont
\textbf{Piosenka turystyczna w starym stylu} \\
\fontsize{8pt}{10pt}\selectfont
Grupa R / sł. i muz. Wiesław Jarosz \\ \\
\fontsize{10pt}{12pt}\selectfont
\leftskip0cm
\begin{tabular}{@{}p{7.50cm}p{3cm}@{}}
\noindent
Poniedziałek trąci nudą & C G \\
Poświąteczne leczy sprawy & E a \\
Taki nieszczególny dzień & F C B G \\
Z wtorkiem włóczy się po mieście & C G \\
Tygodniowy bagaż niesie & A D7 \\
Słońca szuka, śmieszny jest. & F C G C (C7) \\ \\
\end{tabular}

\leftskip1cm
\noindent
\begin{tabular}{@{}p{6.50cm}p{3cm}@{}} 
A w sobotę, a w sobotę & F G \\
Wezmę plecak i gitarę & C a \\
Znów przywitam góry, las & F G \\
I rajdową starą wiarę & C C7 \\
A w sobotę, a w sobotę & F G \\
Wezmę plecak i gitarę & C a \\
Znów przywitam góry, las & F G \\
Znów przywitam góry, las & F C \\ \\
\end{tabular}

\leftskip0cm
\noindent
\begin{tabular}{@{}p{7.50cm}p{3cm}@{}}
Środa gwiżdże na to wszystko \\
Święci z nieba leją wodę \\
Jeszcze tylko doby dwie \\
Czwartek poczuł już przygodę \\
Pod parasol bierze środę \\
Pójdą z nami, dobrze wiem \\ \\
\end{tabular}

\leftskip1cm
\noindent
\begin{tabular}{@{}p{7.50cm}p{3cm}@{}}
A w sobotę… \\ \\
\end{tabular}

\leftskip0cm
\noindent
\begin{tabular}{@{}p{7.50cm}p{3cm}@{}}
Piątek stroi się od rana \\
A dziewczyna czas rozstania \\
Opłakuje, to nic, że \\
Że w kieszeni ciągle mało \\
Że nas słońce znów nabrało \\
Gwizd pociągu wezwie mnie. \\ \\
\end{tabular}

\leftskip1cm
\noindent
\begin{tabular}{@{}p{7.50cm}p{3cm}@{}}
A w sobotę... \\
I piosenkę turystyczną \\
Mam ich ze sobą cały kram \\
Tę, którą właśnie śpiewam wam
\end{tabular}

\end{document}
